\documentclass[12pt]{article}
\usepackage[a4paper, top=15mm, left=15mm, right=10mm, bottom = 15mm]{geometry}
\usepackage{amsmath, amssymb}
\usepackage{enumerate}
\usepackage{listings, color}

\title{Chap 6 solutions}
\author{Huy Nguyen}
\newcommand{\co}{\texttt}
\renewcommand{\and}{\ \&\& \ }

\lstset{
  backgroundcolor = \color{white},
  frame=tb,
  language=C,
  aboveskip=3mm,
  belowskip=3mm,
  showstringspaces=false,
  columns=flexible,
  basicstyle={\small\ttfamily},
  numbers=none,
  numberstyle=\tiny\color{gray},
  keywordstyle=\color{blue},
  otherkeywords = {obj, SEL, @, BOOL},
  commentstyle=\color{red},
  stringstyle=\color{mauve},
  breaklines=true,
  breakatwhitespace=true,
  tabsize=3
}

\begin{document}

\section{Problem 6.2}
Let $b[i]$ be the minimum total penalty for stopping at hotel $a_i$, $1 \le i \le n$. We have $$b[i] = \min_{1 \le j < i} \{ b[j] + (200 - (a[i] - a[j]))^2\}.$$
Also record the value $j$ which yields $\displaystyle \min_{1 \le j < i} \{ b[j] + (200 - (a[i] - a[j]))^2\}$ and set \co{b[i].prev = b[j]}. Backtrack from $b[n]$ to get the sequence of hotels to stop by.

\section{Problem 6.3}
Let $S[i]$ be the maximum total profit we get from building some restaurants in $\{m_1, m_2, \ldots, m_i\}$. Consider 2 cases:
\begin{enumerate}[(1)]
	\item If restaurant $m_i$ should not be built, then $S[i] = S[i-1]$.
	\item If restaurant $m_i$ should be built, then let $c_i$ be the maximum index $j$ which yields $m_i - m_j \ge k$. We then have $\displaystyle S[i] = p_i + S[c_i]$.
\end{enumerate}
Therefore in general, $$S_i = \max \{ S[i-1], p_i + S[c_i]\}.$$
To get the sequence of restaurants, keep an array $R[n]$ such that $R[i] = 1$ if restaurant $i$ is built in the optimal solution, and $R[i] = 0$ otherwise. When we calculate $S[i]$, if the max falls to case (1), $R[i] = 0$. Else, $R[i] = 1$. Output all the $R[i]$s that are $1$.

\section{Problem 6.4}
Consider an array $S[n]$ where $S[i] = \co{true}$ if the substring $s_1 s_2 \ldots s_i$ is a valid string, and \co{false} otherwise. We have $S[1] = \co{dict(s[1])}$ and 
$$
\begin{aligned}
S[i] = & \ (S[1] \and \co{dict(s[2 .. i]}) \\ 
& || (S[2] \and \co{dict(s[3..i])} \\ 
& || \ldots \\
& || (S[i-1] \and \co{dict(s[i..i])})), 
\end{aligned}$$
where $s[j..i]$ is $s_j s_{j+1} \ldots s_i$.

\section{Problem 6.6}
Let the input string be $x_1 x_2 \ldots x_n$. \\
Let $Z = \{a,b,c\}$ and let $T[i,j] \subset Z$ be the set of the possible values that the product $x_i x_{i+1} \ldots x_{j}$ can yield with all possible parenthesizations. \\
We see that $T[i,i] = x_i$ for all $1 \le i \le n$. We need to compute $T[1,n]$. \\
Define $A \times B$ as $\{ a \cdot b | a \in A, b \in B\}$. \\
We note that $T[i, i+1] = T[i,i] \bigcup T[i+1, i+1]$ and $T[i,i+2] = (T[i,i] \times T[i+1, i+2]) \bigcup (T[i,i+1] \times T[i+2,i+2])$ (to put it another way, $abc$ can be written as $(a)(bc)$ or $(ab)(c)$). \\
We therefore see that we already have $$T[1,1], T[2,2], T[3,3], \ldots,$$
from which we can calculate $$T[1,2], T[2,3], T[3,4], \ldots,$$
from which we can calculate $$T[1,3], T[2,4], T[3,5], \ldots $$
and eventually we can expand to $T[1,n]$, which is what we need to find.\\
In other words, $$T[i, i + s] = \bigcup_{i \le k < i + s} (T[i,k] \times T[k+1, i + s]).$$
The algorithm is as follows:
\begin{lstlisting}
	for i = 1 to n: T[i,i] = x[i].
	for s = 1 to n-1:
		for i = 1 to n - s:
			T[i, i + s] = empty
			for k = 1 to i + s - 1:
				T[i, i + s] = T[i, i + s] UNION (T[i, k] * T[k+1,s])
	If a is in T[1,n] return true. Else, return false.
\end{lstlisting}

\section{Problem 6.7}
Let the input string be $x_1 x_2 \ldots x_n$. Let $T[i,j]$ be the length of the longest palindromic subsequence in $x[i..j]$. We have $$\begin{cases} T[i.i] = 1 \\ T[i,i+1] = 2 \text{ if } x[i] = x[i+1] \text{ and } 0 \text{ if } x[i] \ne x[i+1] \\ T[i,j] = T[i+1,j-1] + 2 \text{ if } x[i] = x[j] \text{ and } \max \{T[i+1,j], T[i,j-1]\} \text{ else} \end{cases}$$

\section{Problem 6.8}
Let $E[i,j]$ be the length of the largest common substring of $x_1 x_2 \ldots x_i$ and $y_1 y_2 \ldots y_j$ such that $x_i = y_j$. We see that $E[1,j] = 1$ if $x_1 = y_j$ and $0$ otherwise. Similarly, $E[j,1] = 1$ if $y_1 = x_j$ and $0$ otherwise.\\
In general, we have $$E[i,j] = \begin{cases} E[i-1,j-1] + 1 \text{ if } x_i = y_j \\ 0 \text{ if } x_i \ne y_j \end{cases}$$

\section{Problem 6.9}
Let the input string be $x[0..n-1]$ and the input breakpoint array be $y[1..m]$. Convert $y$ to $y[0..m+1]$ and let $y[0] = -1, y[m+1] = n-1$. \\
Let $M(i,j)$ be 
$$\begin{cases} M(i,i)=0\text{, }\forall i: 0 \le i \le m + 1 \\
M(i,i + 1)=0\text{, }\forall i: 0 \le i \le m + 1\\
M(i,j)=\left(y[j]-y[i]\right) + \displaystyle \min_{l: i < l < j}\{M(i,l) + M(l,j)\}
\end{cases}$$

\section{Problem 6.10}
Let $E[i,j]$ be the probability of obtaining exactly $i$ heads when $j$ coins $c_1, c_2, \ldots, c_j$ with head-probability $p_1, p_2, \ldots, p_j$ are tossed.
We have $E[0,j] = (1-p_1)(1-p_2) \ldots (1-p_j)$ and $$E[i,j] = 1 - \sum_{k=1}^i E[i-k,j].$$

\section{Problem 6.11}
Let $E[i,j]$ be the longest common subsequence of $x_1 x_2 \ldots x_j$ and $y_1 y_2 \ldots y_j$. \\
We have $E[1,j] = 1$ if $x_1 = y_j$, for all $1 \le j \le m$. Similarly, $E[j,1] = 1$ if $x_j = y_1$, for all $1 \le j \le n$. \\
In general we have $$E[i,j] = \begin{cases} E[i-1,j-1] + 1 \text{ if } x_i = y_j \\ \max \{E[i-1,j] + E[i,j-1] \} \text{ if } x_i \ne y_j. \end{cases}$$

\section{Problem 6.12}
Let $d[i,j]$ be the distance between point $i$ and $j$. We have $$\begin{cases} A[i,i] = 0 \\ A[i, i + 1] = 0 \\ A[i, i+2] = 0 \\ A[i,j] = \displaystyle \min_{i < k < j} \{ A[i,k] + A[k,j] + d[i,k] + d[k,j]\} \end{cases}$$

\section{Problem 6.13}
Let $E[i,j]$ be the maximum value the first player can have by picking cards from the set of cards $s_i, s_{i+1}, \ldots, s_j$. \\
We see that $E[i,j] = 0$ if $i \le j$. In general, $$\begin{aligned} E[i,j] = \max\{ & v_i + \min \{ E[i + 2, j] - v_{i+1}, E[i + 1, j - 1] - v_j \}, \\ & v_j + \min \{ E[i+1,j-2] - v_{j-1}, E[i+2,j-1] - v_{i+1}\}\}. \end{aligned}$$
\end{document}